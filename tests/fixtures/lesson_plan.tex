% Compile document with LuaLaTeX
\documentclass[12pt]{article}

\usepackage{fontspec}
\usepackage{unicode-math}
\setromanfont{Source Serif 4}
\setsansfont{Source Sans 3}
\setmonofont{Source Code Pro}
\setmathfont{Latin Modern Math}
\setmainfont{Source Sans 3}
\usepackage[french]{babel}
\usepackage[letterpaper, margin=1in]{geometry}
\usepackage{amsmath}
\usepackage{enumitem}
\usepackage{xltabular}
\usepackage{booktabs}
\usepackage{fontawesome5}

\title{Plan de leçon\\ Révision des fonctions exponentielles et des logarithmes}
\author{Vincent Archambault}
\date{}

\begin{document}

\maketitle

\section*{Informations générales}
\begin{description}
\item[\faBook{} Cours] Calcul différentiel
\item[{\faHashtag} Numéro] 11
\item[{\faCalendar*[regular]} Date] 6 novembre 2024
\item[{\faClock[regular]} Heure] 15h30 - 16h45 (1 heure 15 minutes)
\item[\faLandmark{} Lieu] Le meilleur cégep
\item[\faBullseye{} Objectifs]
\mbox{}\newline\leavevmode\vspace{-3ex}\begin{itemize}
    \item Réviser les notions du secondaire sur les exponentielles et les logarithmes.
    \item Appliquer les propriétés des exponentielles et des logarithmes pour résoudre ou simplifier des équations.
\end{itemize}
\item[\faToolbox{} Compétences du devis] Analyser des problèmes par l’application du calcul différentiel.
\item[\faUserCheck{} Critères de performance du devis] Utilisation pertinente du langage et des concepts dans l’application du calcul différentiel.
\item[\faTv{} Matériel nécessaire] Ordinateur, projecteur et tableau.
\end{description}

\clearpage
\section*{Résumé de la leçon}
\begin{description}
\item[Durée totale] 1 heure 15 minutes
\end{description}
\renewcommand{\arraystretch}{1.5}
\newcolumntype{Y}{>{\raggedright\arraybackslash}X}
\newcolumntype{Z}{>{\raggedright\arraybackslash}p{2cm}}
\begin{xltabular}{\textwidth}{>{\hsize=.4\hsize}Y >{\hsize=.6\hsize}Y Z}
\toprule
\textbf{Partie} & \textbf{Objectif} & \textbf{Durée} \\ \midrule \endfirsthead
\toprule
\textbf{Partie} & \textbf{Objectif} & \textbf{Durée} \\ \midrule \endhead
Introduction & & 8 min \\
Fonctions transcendantes & Nommer des fonctions transcendantes et leurs applications. & 2 min \\ 
Exponentielles : définitions et propriétés & Résoudre et simplifier des équations exponentielles. & 20 min \\ 
Exponentielles : graphe & Tracer et décrire le graphique d'une fonction exponentielle. & 10 min \\ 
Logarithmes : définition et propriétés & Résoudre et simplifier des équations logarithmiques. & 20 min \\ 
Logarithmes : graphe & Tracer et décrire le graphique d'une fonction logarithmique. & 10 min \\ 

Conclusion &  & 5 min \\
\bottomrule
\end{xltabular}

\clearpage
\section{Introduction}
\noindent 8 minutes, début à 15h30 et fin à 15h38
\subsubsection*{\faBullseye{} Objectifs de la leçon}
\begin{itemize}
    \item Réviser les notions du secondaire sur les exponentielles et les logarithmes.
    \item Appliquer les propriétés des exponentielles et des logarithmes pour résoudre ou simplifier des équations.
\end{itemize}
\subsubsection*{\faList{} Plan de la leçon}
\begin{itemize}
    \item Fonctions transcendantes
    \item Exponentielles : définitions et propriétés
    \item Exponentielles : graphe
    \item Logarithmes : définition et propriétés
    \item Logarithmes : graphe
\end{itemize}
\subsubsection*{\faSurprise[regular]{} Amorçage}
Plier une feuille de papier 42 fois pour atteindre la Lune.

\clearpage
\section{Fonctions transcendantes}
\noindent 2 minutes, début à 15h38 et fin à 15h40
\subsubsection*{\faBullseye{} Objectif}
Nommer des fonctions transcendantes et leurs applications.
\subsubsection*{\faChalkboardTeacher{} Activité d'enseignement}
Mentionner qu'il y a d'autres fonctions que les fonctions polynomiales. Notamment pour ce cours les fonctions exponentielles, les fonctions logarithmiques et les fonctions trigonométriques.
\subsubsection*{\faCalculator{} Activité d'apprentissage}
Nommer des applications de fonctions transcendantes. Exemples: 
\begin{itemize}
\item croissance d'une population
\item décroissance d'une substance radioactive
\item taux de croissance d'un investissement.
\end{itemize}



\clearpage
\section{Exponentielles : définitions et propriétés}
\noindent 20 minutes, début à 15h40 et fin à 16h00
\subsubsection*{\faBullseye{} Objectif}
Résoudre et simplifier des équations exponentielles.
\subsubsection*{\faChalkboardTeacher{} Activité d'enseignement}
Expliquer l'exponentielle comme une multiplication répétée.
\begin{itemize}
\item Nommer les différentes parties: base, exposant. Base $a > 0$ et $a \neq 1$.
\item Dériver les propriétés des exponentielles comme une multiplication répétée.
\item Exemple 1 section 3.2.
\end{itemize}

\subsubsection*{\faCalculator{} Activité d'apprentissage}
Testez votre compréhension section 3.2.


\clearpage
\section{Exponentielles : graphe}
\noindent 10 minutes, début à 16h00 et fin à 16h10
\subsubsection*{\faBullseye{} Objectif}
Tracer et décrire le graphique d'une fonction exponentielle.
\subsubsection*{\faChalkboardTeacher{} Activité d'enseignement}
Nommer les propriétés du graphique d'une fonction exponentielle: 
\begin{itemize}
\item passe par $(0,1)$.
\item strictement croissante si $a > 1$ et strictement décroissante si $0 < a < 1$.
\item Le domaine est $(-\infty, \infty)$ et l'image est $(0, \infty)$.
\end{itemize}

\subsubsection*{\faCalculator{} Activité d'apprentissage}
Répondre aux questions de l'enseignant


\clearpage
\section{Logarithmes : définition et propriétés}
\noindent 20 minutes, début à 16h10 et fin à 16h30
\subsubsection*{\faBullseye{} Objectif}
Résoudre et simplifier des équations logarithmiques.
\subsubsection*{\faChalkboardTeacher{} Activité d'enseignement}
\begin{itemize}
    \item Expliquer que le logarithme est l'inverse de l'exponentielle.
    \item Définir le logarithme en tant que fonction.
    \item Propriétés des logarithmes.
\end{itemize}
\subsubsection*{\faCalculator{} Activité d'apprentissage}
Testez votre compréhension section 3.2.


\clearpage
\section{Logarithmes : graphe}
\noindent 10 minutes, début à 16h30 et fin à 16h40
\subsubsection*{\faBullseye{} Objectif}
Tracer et décrire le graphique d'une fonction logarithmique.
\subsubsection*{\faChalkboardTeacher{} Activité d'enseignement}
Nommer les propriétés du graphique d'une fonction logarithmique:
\begin{itemize}
\item passe par $(1,0)$.
\item strictement croissante si $a > 1$ et strictement décroissante si $0 < a < 1$.
\item Le domaine est $(0, \infty)$ et l'image est $(-\infty, \infty)$.
\end{itemize}

\subsubsection*{\faCalculator{} Activité d'apprentissage}
Répondre aux questions de l'enseignant

\clearpage
\section{Conclusion}
\noindent 5 minutes, début à 16h40 et fin à 16h45
\subsubsection*{\faList{} Résumé de la leçon}
\begin{itemize}
    \item Réviser les notions du secondaire sur les exponentielles et les logarithmes.
    \item Appliquer les propriétés des exponentielles et des logarithmes pour résoudre ou simplifier des équations.
\end{itemize}
\subsubsection*{\faCalendar*[regular]{} Pour la prochaine fois}
Chapitre 3.14, exercices 2.718

\end{document}